\documentclass[a4paper]{article}

\usepackage[french]{babel}
\usepackage[utf8]{inputenc}
\usepackage[T1]{fontenc}
\usepackage{lmodern}
\usepackage[babel]{microtype}
\usepackage{lipsum}

\author{Pierre Hauweele}
\title{Rapport de Projet d'Informatique}
\date{2019--2020}

\begin{document}
\maketitle

\section{Introduction}
\lipsum[1-1]

\section{Guide d'utilisation}
\subsection{Mode graphique}
On lance le jeu avec
\begin{verbatim}
gradle run
\end{verbatim}
\subsection{Mode statistiques}
Pour ce projet, le mode statistiques se lance avec la commande
\begin{verbatim}
gradle runStat
\end{verbatim}

\section{Implémentation}
Dans le répertoire \texttt{misc/} vous pouvez trouver les fichiers que
j'ai choisi d'ajouter à mon archive. Ces fichiers ne sont pas demandés
par l'énoncé.
\begin{itemize}
  \item Le dossier \texttt{code\_rapport/}: ce sont les sources
    \LaTeX\footnote{\LaTeX\ est un système de composition de documents
    fonctionnant de manière programmative.}\ de mon rapport.
  \item Le fichier \texttt{build\_archive.sh}: il s'agit du script
    Bash\footnote{Bash est interpréteur de scripts souvent disponible sur
    Linux.} que j'utilise pour générer mon archive à rendre. Il teste le
    projet, nettoie les répertoires et crée l'archive.
\end{itemize}

\end{document}
